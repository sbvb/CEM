% POLI PF PGTO A VISTA
%
\documentclass[a4paper,7.5pt]{article}
\usepackage[brazil]{babel}
\usepackage[utf8x]{inputenc}
\usepackage{fancyhdr}
\usepackage{lastpage}
\usepackage{fix-cm}
\usepackage{graphicx}
\usepackage{ragged2e}
\usepackage[pdftex]{hyperref}

\renewcommand{\rmdefault}{phv} % Arial
\renewcommand{\sfdefault}{phv} % Arial

\setlength{\paperwidth}{21cm}
\setlength{\paperheight}{29.7cm}
\addtolength{\hoffset}{-0.54cm}
\addtolength{\voffset}{-1.54cm} 
\setlength{\oddsidemargin}{0cm} 
\setlength{\evensidemargin}{0cm}
\setlength{\headheight}{1.5cm}
\setlength{\headsep}{0.5cm}
\setlength{\marginparsep}{0cm}
\setlength{\topmargin}{0cm}
\setlength{\textheight}{25cm}
\setlength{\textwidth}{17.5cm}

\pagestyle{fancy}
\begin{document}
\lhead{
    \parbox{4cm}{
        \includegraphics[width=40mm]{polilogo.jpg}}
}
\chead{
    \parbox{10cm}{
    	\begin{center}
            \fontsize{14}{12}\selectfont
    		\bf CONTRATO DE PRESTAÇÃO \\DE SERVIÇOS EDUCACIONAIS
    	\end{center}
    }
}
\rhead{
    \parbox{4cm}{
    	\begin{center}
            \fontsize{10}{12}\selectfont
    		\bf POLI – ::NumdeProjeto:::
    	\end{center}
    }
}
\lfoot{}
\cfoot{}
\rfoot{\thepage/\pageref{LastPage}}
\renewcommand{\headrulewidth}{0.4pt}
\renewcommand{\footrulewidth}{0.4pt}

\begin{flushright}
	\parbox{9.5cm}{    	
           	 \fontsize{7.5}{12}\selectfont
    			\bf{CONTRATO DE PRESTAÇÃO DE SERVIÇO EDUCACIONAL QUE CELEBRAM ENTRE SI A FUNDAÇÃO COPPETEC E ::Nome do(a) discente::: MEDIANTE AS CONDIÇÕES E CLÁUSULAS ABAIXO.}
	}
\end{flushright}

\fontsize{7.5}{9}\selectfont
\justifying
\noindent {\bf CONTRATANTE: ::Nome do(a) discente:::}, ::Estado\_civil:::, ::Profissão:::, CPF nº. ::CPF do(a) discente:::, carteira de identidade nº. ::Identidade do(a) discente::: – ::Orgão\_Expedidor da identidade do(a) discente:::, residente e domiciliado(a) na ::endereço do(a) discente, incluindo bairro e cidade:::, Rio de Janeiro, RJ, CEP nº.  ::CEP da residência do(a) discente:::, telefone de contato {\bf::telefone de contato do(a) discente com DDD:::}, celular ::número do celular do(a) discente, com DDD:::, e-mail de contato ::email do(a) discente:::, e-mail para envio de Nota Fiscal Eletrônica ::email do(a) discente para envio de NF eletrônica:::.
\\\\
{\bf CONTRATADA: FUNDAÇÃO COORDENAÇÃO DE PROJETOS, PESQUISAS E ESTUDOS TECNOLÓGICOS - COPPETEC} privada sem fins lucrativos, instituída em 12/03/93 conforme escritura registrada sob o nº de ordem 125.161 do livro “A” nº 33 do Registro Civil das Pessoas Jurídicas em 24/03/93, inscrita no CNPJ / MF sob o nº 72.060.999/0001-75, Inscrição Municipal nº 01.119.923, com sede no Centro de Gestão Tecnológica da COPPE/UFRJ – CGETEC – CT2, Avenida Moniz Aragão, s/nº, Cidade Universitária da UFRJ, Ilha do Fundão, Rio de Janeiro, RJ, Brasil, CEP 21.941-972", neste ato representado por seu Diretor Superintendente, Segen Farid Estefen.
\\\\
{\bf CLÁUSULA PRIMEIRA}
\\\\
\indent Os serviços decorrentes do presente Contrato, a serem prestados pela {\bf CONTRATADA} ao(à) {\bf CONTRATANTE}, serão executados em ação coordenada com a {\bf Escola Politécnica da Universidade Federal do Rio de Janeiro (POLI/UFRJ)}, cujos meios, reCursos e estrutura serão utilizados para sua realização.
\\\\
 {\bf CLÁUSULA SEGUNDA - OBJETO}
\\\\
{\bf 2.1-}\indent O presente Contrato tem por objetivo a prestação de serviços educacionais pela {\bf CONTRATADA}, uma vaga, para a realização do {\bf Curso ::Nome\_específico\_do\_Curso citando-se 'Curso de Especialização, Pós-Graduação Lato Sensu, xyz' ou 'Curso de Extensão xyz'::: – ::Sigla do Curso, quando houver::: – (::Turma:::)}, com carga horária total discente de {\bf ::Total de Horas:::} horas, pela {\bf Escola Politécnica da UFRJ}, às {\bf ::dias da semana (segunda-feira, terça-feira, etc) de realização do Curso:::}, com horário de início às {\bf ::Hora de Início NNhNNm:::} e término às {\bf::Hora de Término NNhNNm:::}, sendo realizado no {\bf::endereço/ local do Curso:::}, Rio de Janeiro, RJ.
\\\\
{\bf2.2-}\indent A Coordenação Geral do Curso será de responsabilidade do Prof. ::Nome do(a) docente:::, CPF nº. ::CPF do(a) docente:::, carteira de identidade nº. ::Identidade do(a) docente::: – ::Orgão::Expedidor da identidade do(a) docente:::, telefone de contato na UFRJ {\bf::telefone de contato do(a) docente da UFRJ com DDD:::}, e-mail ::email do(a) docente na Politécnica:::.
\\\\
{\bf2.3-}\indent Na ocorrência de qualquer impedimento à execução do programa estabelecido, a {\bf CONTRATADA} e/ou a {\bf POLI/UFRJ} poderão antecipar ou prorrogar o prazo de duração do Curso, bem como a data de seu início e de seu término, e o horário das aulas, podendo, inclusive, alterar os dias de aula, ou, ainda, fixar outro dia da semana para aulas, quando ocorrerem feriados coincidentes com o dia de aula pré-estabelecido nessa cláusula, podendo a seu critério utilizar outras instalações para a realização do Curso. A {\bf CONTRATADA} e/ou {\bf POLI/UFRJ} se comprometem a comunicar ao(à) {\bf CONTRATANTE}, com antecedência mínima de {\bf 24} (vinte e quatro) {\bf horas}, qualquer modificação do local de realização do Curso.
\\\\
{\bf Parágrafo único:}\indent A {\bf CONTRATADA} e/ou a {\bf POLI/UFRJ} poderão, se necessário, introduzir, a qualquer momento, modificações no Programa do \indent\indent\indent\indent\indent Curso, desde que não afetem substancialmente seus objetivos básicos.
\\\\
{\bf CLÁUSULA TERCEIRA – DO VALOR}
\\\\
{\bf3.1-}\indent Como contraprestação aos serviços prestados e disponibilizados, o(a) {\bf CONTRATANTE} pagará à {\bf CONTRATADA o valor de R\$ ::Valor\_global\_do\_Curso::: (reais)}.
\\\\
{\bf3.2-}\indent O pagamento será realizado em parcela única, após a assinatura do presente contrato, mediante o recebimento do respectivo documento de cobrança.
\\\\
{\bf3.3-}\indent Em caso da falta de pagamento na data do vencimento prevista no documento de cobrança, o valor previsto será acrescido dos seguintes encargos: {\bf multa} de {\bf 02\%} (dois por cento); {\bf juros de mora de 01\%} (um por cento) {\bf ao mês} e {\bf atualização monetária}, se houver, de {\bf0,5\%} (meio por cento) {\bf ao mês} até a data da sua efetiva quitação.
\\\\
{\bf CLÁUSULA QUARTA - RESCISÃO}
\\\\
{\bf4.1-}\indent À {\bf CONTRATADA} fica assegurado o direito de rescindir o presente contrato mediante simples aviso com aceite pelo(a) {\bf CONTRATANTE} ou, na sua impossibilidade, por {\bf AR-Aviso de Recebimento}, na hipótese do(a) {\bf CONTRATANTE} não se adequar aos princípios norteadores da {\bf CONTRATADA} e da {\bf Escola Politécnica da UFRJ}, ou comprometer o seu nome, ou sua reputação, ou praticar atos de indisciplina.
\\\\
{\bf4.2-}\indent O presente contrato poderá ser considerado rescindido de pleno direito, sem necessidade de notificação judicial ou extrajudicial, em razão de ser descumprimento culposo ou infração de qualquer de suas Clausulas ou condições pela outra parte, respondendo ainda por perdas e danos, mais juros e atualização monetária, segundos os índices oficiais regularmente estabelecidos e honorários advocatícios.
\\\\
{\bf4.3-}\indent O simples fato do(a) CONTRATANTE deixar de comparecer às aulas ministradas pela CONTRATADA não rescinde o presente Contrato.
\\\\
{\bf CLÁUSULA QUINTA - DESLIGAMENTO}
\\\\
{\bf5.1-}\indent O(A) {\bf CONTRATANTE} poderá solicitar o {\bf desligamento do Curso}, sempre por meio de requerimento escrito.
\\\\
{\bf5.2-}\indent Na hipótese da {\bf desistência} ser comunicada, com antecedência mínima de {\bf15} (quinze) dias, contados da data de início do Curso, a {\bf CONTRATADA} restituirá ao(a) {\bf CONTRATANTE} o valor total dos serviços educacionais {\bf pago}, deduzido a importância correspondente a {\bf15\% }(quinze por cento), a título de ressarcimento pelos custos pré-estabelecidos necessários para a realização do Curso.
\\\\
{\bf5.2.1-}\indent O(A) {\bf CONTRATANTE} declara estar {\bf ciente} de que adere, neste ato, a um {\bf grupo fechado}, o qual, uma vez constituído, tem seus {\bf custos pré-estabelecidos}, o que impossibilita a isenção total do valor referente a estes custos, considerando a manutenção e o equilíbrio econômico-financeiro do Curso.
\\\\
{\bf CLÁUSULA SEXTA – DOS MATERIAIS}
\\\\
{\bf6.1-}\indent Fica expressamente entendido que todos os materiais utilizados e/ou distribuídos durante a realização ou para a realização do {\bf Curso ::Nome\_específico\_do\_Curso citando se 'Curso de Especialização, Pós-Graduação Lato Sensu, xyz' ou 'Curso de Extensão xyz'::: – ::Sigla do Curso, quando houver::: – (::Turma:::)} destinam-se à utilização exclusiva do(a) {\bf CONTRATANTE}, não podendo reproduzi-los ou de qualquer forma utilizá-los {\bf sem autorização} por escrito pela {\bf Coordenação do Curso.}
\\\\
{\bf6.2-}\indent O material didático, no decorrer do Curso, será fornecido pela {\bf CONTRATADA}, preferencialmente, mas não necessariamente, através de mídia eletrônica ou Internet. Por critério exclusivo da {\bf CONTRATADA}, o material didático poderá ser disponibilizado em mídia impressa.
\\\\
{\bf6.3-}\indent Fica expressamente proibida a gravação de áudio e/ou imagem, bem como sua distribuição, das aulas, orientações, palestras e outros eventos associados ao Curso, salva autorização por escrito da coordenação a partir de solicitação, também por escrito do(a) discente.
\\\\
{\bf Parágrafo Único:}\indent Todo material extra (livros, fotocópias, artigos etc.) indicado pelos professores deverá ser adquirido e custeado exclusivamente pelo(a) {\bf CONTRATANTE}.
\\\\
{\bf CLÁUSULA SÉTIMA – DA AVALIAÇÃO E APROVAÇÃO}
\\\\
{\bf7.1-}\indent O(a) discente inscrito(a) na vaga contratada pelo(a) {\bf CONTRATANTE} somente receberá o Certificado de conclusão {\bf ::Nome\_específico\_do\_Curso citando-se 'Curso de Especialização, Pós-Graduação Lato Sensu, xyz' ou 'Curso de Extensão xyz'::: – ::Sigla do Curso, quando houver::: – (::Turma:::)} se, cumulativamente:
\begin{itemize}
 \setlength{\itemsep}{1pt}
  \setlength{\parskip}{0pt}
  \setlength{\parsep}{0pt}
\item Atender às Normas Específicas contidas no Manual do(a) Discente, normas informadas no momento ou previamente à assinatura deste instrumento contratual e Manual do(a) Discente entregue no primeiro dia de aula ao {\bf CONTRATANTE};
\item Atender aos regulamentos da UFRJ, afetos à realização do Curso, disponíveis no site da UFRJ (\url{www.ufrj.br/pr-2}, link CEPG e/ou outros);
\item For aprovado em todas as disciplinas constantes do Curso e na monografia (ou Projeto de Fim de Curso – PFC ou avaliações atinentes). O aproveitamento do(a) discente será expresso mediante os seguintes conceitos: A = Excelente, B = Bom, C = Regular e D = Deficiente. Serão considerados aprovados em uma disciplina ou na monografia o(a)s discentes avaliado(a)s com os conceitos A, B ou C;
\item For aprovado na Monografia (ou Projeto de Fim de Curso – PFC ou avaliações atinentes), perante banca examinadora e depositar os exemplares, conforme Normas Específicas contidas no Manual do(a) Discente, dentro do prazo estabelecido pela Coordenação;
\item Obtiver Coeficiente de Rendimento Acumulado (CRA>=2,1). O CRA será calculado pela média ponderada dos conceitos, a que serão atribuídos os valores A = 3, B = 2; C = 1; D = 0; e
\item Estiver quite com todas as parcelas de pagamento, conforme Cláusula Terceira.
\end{itemize}
{\bf CLAUSULA OITAVA – DO CERTIFICADO}
\\\\
\indent Ao(À) discente que cumprir todas as exigências para aprovação, será oferecido Certificado de {\bf ::Especialista em ou de Curso de Extensão::: ::Nome\_específico\_do\_Curso:::}, emitido pela Universidade Federal do Rio de Janeiro.
\\\\
{\bf CLAUSULA NONA – VIGÊNCIA}
\\\\
\indent A vigência do presente Contrato será de {\bf ::número de meses do contrato::: (::número de meses do contrato por extenso:::)} meses e coincidirá com o período de aulas do Curso e com entrega dos trabalhos de conclusão.
\\\\
{\bf CLÀUSULA DÉCIMA – DO FORO}
\\\\
\indent Fica eleito o {\bf Foro Seção Judiciária Central da Comarca da Capital do Estado do Rio de Janeiro}, com exclusão expressa de qualquer outro, por mais privilegiado que possa vir a ser, para dirimir qualquer dúvida ou controvérsia decorrente do presente {\bf Contrato}.
\\\\
\indent Assim sendo, as partes firmam o presente instrumento, em {\bf 04} (quatro) {\bf vias} de igual teor e forma, obrigando-se com todas as suas cláusulas, para os devidos efeitos legais.
\\\\
\begin{flushright}
	{\bf Rio de Janeiro, ::Data\_do\_contrato:::}
\end{flushright}
\noindent
\parbox{7.5cm}{    	
	\fontsize{7.5}{9}\selectfont
	\vspace{-0.2cm}
	\hrule
	\centering {\bf::Nome do Discente:::}
	\\
	CONTRATANTE
}
\indent\indent\indent\indent
\parbox{7.5cm}{    	
	\fontsize{7.5}{9}\selectfont
	\vspace{0.5cm}
	\hrule
	\centering {\bf Segen Farid Estefen}
	\\
	Diretor Superintendente
	\\
	FUNDAÇÃO COPPETEC
	\\
	CONTRATADA
}
\\
\parbox{7.5cm}{    	
	\fontsize{7.5}{9}\selectfont
	\vspace{0.5cm}
	\hrule
	\centering {\bf::Nome do Coordenador:::}
	\\
	Coordenador Geral
	\\
	Escola Politécnica/UFRJ
}
\\\\\\\\\\
Testemunhas:
\\
\parbox{7.5cm}{    	
	\fontsize{7.5}{9}\selectfont
	\vspace{1.7cm}
	\hrule
	\centering {\bf::Testemunha da Escola Politécnica:::}
	\\
	Escola Politécnica / UFRJ	
}
\indent\indent\indent\indent
\parbox{7.5cm}{    	
	\fontsize{7.5}{11}\selectfont
	\vspace{1.5cm}
	\hrule
	\centering {\bf :: Testemunha – preferencialmente – da(o) contratante:::}	
}
\end{document}

